\documentclass[11pt]{beamer}

%\usetheme{Antibes} % Warsaw, Montpellier # nice headers, PaloAlto (Dr Gatendes)
%\usecolortheme{rose}    % Other options: seagull, dolphin, whale, beaver, lily, wolverine (yellowish) etc.

% PaloAlto + dolphin % or orchid good for lectures
% Luebeck + dolphin % good for assignments - simple
% Luebeck + orchid % good for assignments
% Berlin + crane % yellow but good

% \usetheme{PaloAlto} % CambridgeUS, Goettingen (Righ sidebar), Luebeck
% \usecolortheme{dolphin}  % whale or beaver or orchid

\usetheme{PaloAlto} % CambridgeUS, Goettingen (Righ sidebar), Luebeck
\usecolortheme{dolphin}  % whale or beaver or orchid


\usepackage[utf8]{inputenc}
\usepackage{amsmath}
\usepackage{amsfonts}
\usepackage{amssymb}
\author{Joshua Mwangi Maina}
\title{Multiple Regression - Matrices}

\subtitle{SDS6105 - Statistics For Data Science}

\date{\today}

%\usepackage{knitr}
%\setbeamercovered{transparent} 
%\setbeamertemplate{navigation symbols}{} 
%\logo{} 
\institute{UoN, Science and Technology} 
%\date{} 
\subject{SDS6105 - Statistics For Data Science} 

\begin{document}

%\fancyfoot[L]{SDS6105 - Statistics For Data Science}

\begin{frame}
\titlepage
\end{frame}

%\begin{frame}
%\tableofcontents
%\end{frame}

\begin{frame}

In multiple regression, we want to estimate the coefficients $\beta_0$, $\beta_1$, and $\beta_2$ in the equation:
\linebreak 

$y = \beta_0 + \beta_1 X_1 + \beta_2 X_2$
\linebreak 

For a least squares solution, we can set up the problem in matrix form and solve it using linear algebra.

1. Set Up the Matrix Equation
\linebreak 
Let:

\hspace*{15pt}$	\mathbf{y}$ be the vector of observed y values.
	\linebreak 
\hspace*{15pt}$\mathbf{X}$ be the matrix of inputs (including a column of ones for the intercept term).

\end{frame}

\begin{frame}

The model can be represented as:
\linebreak 
\linebreak 
$[\mathbf{y} = \mathbf{X} \cdot \boldsymbol{\beta} + \mathbf{\varepsilon}]$

where:

\hspace*{15pt}$\mathbf{y} = \begin{bmatrix} y_1 \\ y_2 \\ \vdots \\ y_n \end{bmatrix}$
\medskip 
\linebreak 
\hspace*{15pt}$\mathbf{X} = \begin{bmatrix} 1 & X_{1,1} & X_{2,1} \\ 1 & X_{1,2} & X_{2,2} \\ \vdots & \vdots & \vdots \\ 1 & X_{1,n} & X_{2,n} \end{bmatrix}$ (with a leading column of ones for the intercept),	
	
\end{frame}

\begin{frame}
\hspace*{15pt}$\boldsymbol{\beta} = \begin{bmatrix} \beta_0 \\ \beta_1 \\ \beta_2 \end{bmatrix}$,
\medskip 
\linebreak 
\hspace*{15pt}$\mathbf{\varepsilon}$ represents the residuals.

\medskip
To estimate \(\boldsymbol{\beta}\), we minimize the sum of squared residuals by solving:
%\linebreak
\bigskip
\linebreak
\hspace*{15pt}$[\boldsymbol{\hat{\beta}} = (\mathbf{X}^T \mathbf{X})^{-1} \mathbf{X}^T \mathbf{y}]$

\end{frame}

\begin{frame}
2. Construct the Matrices with Your Data

Given:

\hspace*{15pt}	$	 X_1 = [0, 1, 2, 3, 4, 5]$ \linebreak 
\hspace*{15pt}	$	 X_2 = [0, 1, 4, 9, 16, 25] $ \linebreak 
\hspace*{15pt}	$	 y   = [9.1, 7.3, 3.2, 4.6, 4.8, 2.9] $ \linebreak 
	
We’ll create the matrix $\mathbf{X}$ and vector $\mathbf{y}$, and then solve for $\boldsymbol{\beta}$.

\end{frame}

\begin{frame}
3.	Construct the matrix $\mathbf{X}$:
	\linebreak 
\hspace*{15pt}	$\bullet$ 	Add a column of ones for the intercept.
\medskip 
\linebreak 
\hspace*{15pt}	$\bullet$ 	This results in:
\medskip 
\linebreak 
				$\mathbf{X} = \begin{bmatrix}
				1 & 0 & 0 \\
				1 & 1 & 1 \\
				1 & 2 & 4 \\
				1 & 3 & 9 \\
				1 & 4 & 16 \\
				1 & 5 & 25
								\end{bmatrix}$
\end{frame}

\begin{frame}
	4.	Construct the matrix equation:\linebreak 
	\linebreak 
\hspace*{15pt} We have $\mathbf{y} = \begin{bmatrix} 9.1 \\ 7.3 \\ 3.2 \\ 4.6 \\ 4.8 \\ 2.9 \end{bmatrix}$.
	\linebreak
	\linebreak  
	5.	Calculate $(\boldsymbol{\beta})$ using the formula:
$[\boldsymbol{\hat{\beta}} = (\mathbf{X}^T \mathbf{X})^{-1} \mathbf{X}^T \mathbf{y}]$
\end{frame}

\begin{frame}

Step 4: Calculate $\mathbf{X}^T \mathbf{X}$
\medskip 
\linebreak 
\hspace*{15pt}	1.	Transpose the $\mathbf{X} matrix (\mathbf{X}^T)$ by swapping rows with columns.
\medskip 
\linebreak 
\hspace*{15pt}	2.	Multiply $\mathbf{X}^T$ by $\mathbf{X}$ to get a 3x3 matrix.
\bigskip 
\linebreak 
Step 5: Calculate $\mathbf{X}^T \mathbf{y}$

	1.	Multiply the transposed matrix $\mathbf{X}^T$ with $\mathbf{y}$ to get a 3x1 matrix.
\bigskip 
\linebreak  
Step 6: Calculate $(\mathbf{X}^T \mathbf{X})^{-1}$
\medskip 
\linebreak 
	1.	Invert the 3x3 matrix obtained in Step 2 using matrix inversion methods (you can do this in Python, R, or Excel if manual inversion is too complex).
\end{frame}

\begin{frame}

Step 7: Calculate $\boldsymbol{\hat{\beta}}$
\bigskip 
\linebreak 
Multiply $(\mathbf{X}^T \mathbf{X})^{-1} by \mathbf{X}^T \mathbf{y}$ 
\medskip 
\linebreak 
to solve for the coefficient vector $(\boldsymbol{\hat{\beta}})$.
\medskip 
\linebreak 
This will give you values for $\beta_0$, $\beta_1$, and $\beta_2$.
\bigskip 
\linebreak 
Step 6: Predict y when $x_1 = 2$
\medskip 
\linebreak 
To predict $y$ when $x_1 = 2$:
\medskip 
\linebreak 
\hspace*{15pt}	Plug $x_1 = 2$ and $x_2 = 2^2 = 4$ into the regression equation:
\medskip 
\linebreak 
\hspace*{15pt}$\hat{y} = \beta_0 + \beta_1 \cdot 2 + \beta_2 \cdot 4$
\medskip 
\linebreak 

Using the estimated values for $\beta_0$, $\beta_1$, and $\beta_2$, compute $\hat{y}$.

\end{frame}

\end{document}







\begin{frame}

This was a demo for Phill...
\linebreak\linebreak
This is suppossed to be another line
\linebreak
%$ \bullet$ bullet 1
%$ \bullet$ bullet 2
%$ \bulled$ bullet 3

3.	Construct the matrix $\mathbf{X}$:
	
\hspace*{15pt} $\bullet$ Add a column of ones for the intercept. %\linebreak 

\hspace*{15pt} $\bullet$ Add a column of ones for the intercept.

\hspace*{15pt} $\bullet$ Add a column of ones for the intercept.


To find the coefficients  $B_0$ ,  $B_1$ , and  $B_2$  for a multiple linear regression model of the form:

\hspace*{15pt}$\hat{y} \cdot T$

$y = B_0 + B_1 \cdot x_1 + B_2 \cdot x_2$

$\hat y = 8.961 - 2.551 \cdot x_1 + 0.298 \cdot x_2$


\end{frame}


% \end{document}